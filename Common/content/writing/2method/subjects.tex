Appropriate identification of research participants is critical to the science and practice of psychology, particularly for generalizing the findings, making comparisons across replications, and using the evidence in research syntheses and secondary data analyses. If humans participated in the study, report the eligibility and exclusion criteria, including any restrictions based on demographic characteristics.
Describe the sample adequately. Detail the sample's major demographic characteristics, such as age; sex; ethnic and/or racial group; level of education; socioeconomic, generational, or immigrant status; disability status; sexual orientation; gender identity; and language preference as well as important topic-specific characteristics (e.g., achievement level in studies of educational interventions). As a rule, describe the groups as specifically as possible, with particular emphasis on characteristics that may have bearing on the interpretation of results. Often, panicipant characteristics can be important for understanding the nature of the sample and the degree to which results can be generalized. To determine how far the data can be generalized, you may find it useful to identify subgroups: Even when a characteristic is not used in analysis of the data, reporting it may give readers a more complete understanding of the sample and the generalizability of results and may prove useful in meta-analytic studies that incorporate the article's results.

When animals are used, report the genus, species, and strain number or other specific identification, such as the name and location of the supplier and the stock designation. Give the number of animals and the animals' sex, age, weight, and physiological condition.