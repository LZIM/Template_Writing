The Method section describes in detail how the study was conducted, including conceptual and operational definitions of the variables used in the study. Different types of studies will rely on different methodologies; however, a complete description of the methods used enables the reader to evaluate the appropriateness of your methods and the reliability and the validity of your results. It also permits experienced investigators to replicate the study. If your manuscript is an update of an ongoing or earlier study and the method has been published in detail elsewhere, you may refer the reader to that source and simply give a brief synopsis of the method in this section.

%Identify subsections. It is both conventional and expedient to divide the Method section into labeled subsections. These usually include a section with descriptions of the participants or subjects and a section describing the procedures used in the study. The latter section often includes description of (a) any experimental manipulations or interventions used and how they were delivered -- for example, any mechanical apparatus used to deliver them; (b) sampling procedures and sample size and precision; (c) measurement approaches (including the psychometric properties of the instruments used); and (d) the research design. If the design of the study is complex or the stimuli require if detailed description, additional subsections or subheadings to divide the subsections may be warranted to help readers find specific information. in these subsections the information essential to comprehend and replicate the study. Insufficient detail leaves the reader with questions; too much detail burdens the reader with irrelevant information. Consider using appendices and/or a supplemental website for more detailed information.
