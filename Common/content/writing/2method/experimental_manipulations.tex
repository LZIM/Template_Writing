If interventions or experimental manipulations were used in the study, describe their specific content. Include the details of the interventions or manipulations intended for each study condition, including controlgroups (if any), and describe how and when interventions (experimental manipulations) were actually administered.

The description of manipulations or interventions should include several elements. Carefully describe the content of the intervention or specific experimental manipulations. Often, this will involve presenting a brief summary of instructions given to participants. If the instructions are unusual or compose the experimental manipulation, you may present them verbatim in an appendix or in an online supplemental archive. If the text is brief, you may present it in the body of the paper if it does not interfere
with the readability of the report. 

Describe the methods of manipulation and data acquisition. If a mechanical apparatus was used to present stimulus materials or collect data, include in the description of procedures the apparatus model number and manufacturer (when important, as in neuroimaging studies), its key settings or parameters (e.g., pulse settings), and its resolution (e.g., regarding stimulus delivery, recording precision). As with the description of the intervention or experimental manipulation, this material may be presented in the body of the paper, in an appendix, in an online supplemental archive, or as appropriate. When relevant—such as, for example, in the delivery of clinical and educational interventions—the procedures should also contain a description of who delivered the intervention, including their level of professional training and their level of training in the specific intervention. Present the number of deliverers along with the mean, standard deviation, and range of number of individuals or units treated by each deliverer.

Provide information about (a) the setting where the intervention or manipulation was delivered, (b) the quantity and duration of exposure to the intervention or manipulation (i.e., how many sessions, episodes, or events were intended to be delivered and how long they were intended to last), (c) the time span taken for the delivery of the intervention or manipulation to each unit (e.g., would the manipulation delivery be complete in one session, or if participants returned for multiple sessions, how much time passed between the first and last session?), and (d) activities or incentives used to increase compliance.

When an instrument is translated into a language other than the language in which it was developed, describe the specific method of translation (e.g., back-translation, in which a text is translated into another language and then back into the first to ensure that it is equivalent enough that results can be compared).

Provide a description of how participants were grouped during data acquisition (i.e., was the manipulation or intervention administered individual by individual, in small groups, or in intact groupings such as classrooms?). Describe the smallest unit (e.g., indi- viduals, work groups, classes) that was analyzed to assess effects. If the unit used for statistical analysis differed from the unit used to deliver the intervention or manipulation (i.e., was different from the unit of randomization), describe the analytic method used to  account for this (e.g., adjusting the standard error estimates or using multilevel analysis).