%Describe relevant scholarship. Discuss the relevant related literature, but do not feel compelled to include an exhaustive historical account. Assume that the reader is knowledgeable about the basic problem and does not require a complete accounting of its history. A scholarly description of earlier work in the introduction provides a summary of the most recent directly related work and recognizes the priority of the work of others. Citation of and specific credit to relevant earlier works are signs of scientific and scholarly responsibility and are essential for the growth of a cumulative science. In the description of relevant scholarship, also inform readers whether other aspects of this study have been reported on previously and how the current use of the evidence differs from earlier uses. At the same time, cite and reference only works pertinent to the specific issue and not those that are of only tangential or general significance. When summarizing earlier works, avoid nonessential details; instead, emphasize pertinent findings, relevant methodological issues, and major conclusions. Refer the reader to general surveys or research syntheses of the topic if they are available. 
%
%Demonstrate the logical continuity between previous and present work. Develop the problem with enough breadth and clarity to make it generally understood by as wide a professional audience as possible. Do not let the goal of brevity lead you to write a statement intelligible only to the specialist.
