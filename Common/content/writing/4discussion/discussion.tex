After presenting the results, you are in a position to evaluate and interpret their implications, especially with respect to your original hypotheses. Here you will examine, interpret, and qualify the results and draw inferences and conclusions from them. Emphasize any theoretical or practical consequences of the results. (When the discussion is relatively brief and straightforward, some authors prefer to combine it with the Results section, creating a section called Results and Discussion.)

Open the Discussion section with a clear statement of the support or nonsupport for your original hypotheses, distinguished by primary and secondary hypotheses. If hypotheses were not supported, offer post hoc explanations. Similarities and differences between your results and the work of others should be used to contextualize, confirm, and clarify your conclusions. Do not simply reformulate and repeat points already made; each new statement should contribute to your interpretation and to the
reader's understanding of the problem. 

Your interpretation of the results should take into account (a) sources of potential bias and other threats to internal validity, (b) the imprecision of measures, (c) the overall number of tests or overlap among tests, (d) the effect sizes observed, and (e) other limitations or weaknesses of the study. If an intervention is involved, discuss whether it was successful and the mechanism by which it was intended to work (causal pathways) and/or alternative mechanisms. Also, discuss barriers to implementing the intervention or manipulation as well as the fidelity with which the intervention or manipulation was implemented in the study, that is, any differences between the manipulation as planned and as implemented.

Acknowledge the limitations of your research, and address alternative explanations of the results. Discuss the generalizability, or external validity, of the findings. This critical analysis should take into account differences between the target population and the accessed sample. For interventions, discuss characteristics that make them more or less applicable to circumstances not included in the study, how and what outcomes were measured (relative to other measures that might have been used), the length of time to measurement (between the end of the intervention and the measurement of outcomes), incentives, compliance rates, and specific settings involved in the study as well as other contextual issues.

End the Discussion section with a reasoned and justifiable commentary on the importance of your findings. This concluding section may be brief or extensive provided that it is tightly reasoned, self-contained, and not overstated. In this section, you might briefly return to a discussion of why the problem is important (as stated in the introduction); what larger issues, those that transcend the particulars of the subfield, might hinge on the findings; and what propositions are confirmed or disconfirmed by the extrapolation of these findings to such overarching issues.

%You may also consider the following issues:
%% What is the theoretical, clinical, or practical significance of the outcomes, and what is the basis for these interpretations? If the findings are valid and replicable, what real-life psychological phenomena might be explained or modeled by the results? Are applications warranted on the basis of this research?
%% What problems remain unresolved or arise anew because of these findings? The responses to these questions are the core of the contribution of your study and justify why readers both inside and outside your own specialty should attend to the findings. Your readers should receive clear, unambiguous, and direct answers.
